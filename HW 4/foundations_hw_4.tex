\documentclass[10pt]{article} %
\usepackage{fullpage}
\usepackage{graphicx}
\usepackage{graphics}
\usepackage{psfrag}
\usepackage{amsmath,amssymb}
\usepackage{enumerate}
\usepackage{amsthm}

\setlength{\textwidth}{6.5in}
\setlength{\textheight}{9in}

\newcommand{\cP}{\mathcal{P}}
\newcommand{\N}{\mathbb{N}}
\newcommand{\Z}{\mathbb{Z}}
\newcommand{\R}{\mathbb{R}}
\newcommand{\Q}{\mathbb{Q}}
\newcommand{\C}{\mathbb{C}}
\newcommand{\points}[1]{{\it (#1 Points)}}
\newcommand{\tpoints}[1]{{\bf #1 Total points.}}

\title{Math 214 -- Foundations of Mathematics\\
Homework 4\\
{\large{\bf Due February 13, 2014}}}
\date{}
\author{Alexander Powell}


\begin{document}
\maketitle

Solve the following problems. Please remember to use complete sentences
and good grammar.

\begin{enumerate}


\item \points{4} Let $x,y\in \Z$. Prove that if $3\not| x$ and $3\not| y$, then $3 | (x^2-y^2)$.

\bigskip

\noindent {\bf Solution:} 

\begin{proof}
First, it should be stated that $(x^2-y^2) = (x + y)(x - y)$.  Now, if $3\not| x$ and $3\not| y$, then there are $4$ cases that must be proven:
\begin{center} {\bf Case 1:} $x = 3n + 1$ and $y = 3m + 1$

Let $3 | (3n + 1)^2 - (3m + 1)^2$

Then, $((3n + 1) + (3m + 1))((3n + 1) - (3m + 1))$

$ = (3n + 1 + 3m + 1)(3n + 1 - 3m - 1)$

$ = 9n^2 + 9mn + 6n - 9mn - 9m^2 - 6m$

$ = 3(3n^2 - 3m^2 + 2n - 2m)$

Therefore, 3 divides $x^2 - y^2$ in this case.  
\end{center}

\begin{center} {\bf Case 2:} $x = 3n + 1$ and $y = 3m + 2$

Let $3 | (3n + 1)^2 - (3m + 2)^2$

Then, $((3n + 1) + (3m + 2))((3n + 1) - (3m + 2))$

$ = (3n + 3m + 3)(3n + 1 - 3m - 2)$

$ = 9n^2 + 9mn + 9n - 9mn - 9m^2 - 9m - 3n - 3m - 3$

$ = 3(3n^2 + 3n - 3m^2 - 3m - n - m - 1)$

Therefore, 3 divides $(x^2 - y^2)$ in this case.  
\end{center}
\begin{center} {\bf Case 3:} $x = 3n + 2$ and $y = 3m + 1$

Let $3 | (3n + 2)^2 - (3m + 1)^2$

Then, $((3n + 2) + (3m + 1))((3n + 2) - (3m + 1))$

$ = (3n + 3m + 3)(3n - 3m + 1)$

$ = 9n^2 + 9mn + 9n - 9mn - 9m^2 - 9m + 3n + 3m + 3$

$ = 3(3n^2 + 3n - 3m^2 - 3m + n + m + 1)$

Therefore, 3 divides $(x^2 - y^2)$ in this case.  
\end{center}
\begin{center} {\bf Case 4:} $x = 3n + 2$ and $y = 3m + 2$

Let $3 | (3n + 2)^2 - (3m + 2)^2$

Then, $((3n + 2) + (3m + 2))((3n + 2) - (3m + 2))$

$ = (3n + 3m + 4)(3n - 3m)$

$ = 9n^2 + 9mn + 12n - 9mn - 9m^2 - 12m$

$ = 3(3n^2 + 4n - 3m^2 - 4m)$

Therefore, 3 divides $(x^2 - y^2)$ in this case.  
\end{center}

\begin{center} {\bf Therefore}, it is proven that if $3\not| x$ and $3\not| y$, then $3 | (x^2-y^2)$. \end{center}

\end{proof}

\bigskip

\item \points{4} Let $n\in \Z$.  Prove that $2 | (n^4-3)$ if and only if $4 | (n^2+3)$. (Hint: prove $n$ is odd)

\bigskip

\noindent {\bf Solution:}

\begin{proof}
Because of the "if and only if" in the question, it is necessary to prove that if $2 | (n^4-3)$ then $4 | (n^2+3)$ and if $4 | (n^2+3)$ then $2 | (n^4-3)$.  This proof will be divided into two parts.  

\begin{raggedright} {\bf Part 1:} Prove that if $2 | (n^4-3)$ then $4 | (n^2+3)$.  To do this we will introduce the lemma that n is odd.
\end{raggedright}

\begin{center} {\bf Step 1:} Prove that if $2 | (n^4-3)$, then n is odd.  Use contrapositive: If $n$ is even, then $2 \not| (n^4-3)$.   \end{center}
\begin{center}
Let $n$ be even, then $n = 2x, x \in \Z$.  

So, $((2x)^4 - 3)$

$ = 16x^4 - 3$

$ = 2(8x^4 - 3/2)$

Therefore, $2 \not| (n^4-3)$.  
\end{center}

\begin{center} {\bf Step 2:} If $n$ is odd, then $4 | (n^2+3)$.  

Let $n$ be odd, then $n = 2x + 1, x \in \Z$.  

So, $(2x + 1)^2 + 3$

$ = (2x + 1)(2x + 1) + 3$

$ = 4x^2 + 4x + 4$

$ = 4(x^2 + x + 1)$

Therefore, if $n$ is an odd number, then $4 | (n^2+3)$.  
\end{center}


\begin{raggedright} {\bf Part 2:} Next, we need to prove that if $4 | (n^2+3)$ then $2 | (n^4-3)$.  Again, we will use a lemma, $n$ is odd.  
\end{raggedright}

\begin{center} {\bf Step 1:} Use contrapositive: If $n$ is even, then $4 \not| (n^2+3)$

Let $n$ be even, then $n = 2x, x \in \Z$.  

So, $(2x)^2 + 3$

$ = 4x^2 + 3$

$ = 4(x^2 + 3/4)$

Therefore, 4 does not divide $(n^2 + 3)$.  
\end{center}
\begin{center} {\bf Step 2:} If $n$ is odd then $2 | (n^4-3)$.  

Let $n$ be odd, then $n = 2x + 1, x \in \Z$.  

So, $(2x + 1)^4 - 3$

$ = 16x^4 + 32x^3 + 24x^2 + 8x - 2$

$ = 2(8x^4 + 16x^3 + 12x^2 + 4x - 1)$

Therefore, if n is odd then $2 | (n^4-3)$.  
\end{center}
\begin{center} {\bf Therefore}, following parts 1 and 2, we have finally proven that $2 | (n^4-3)$ if and only if $4 | (n^2+3)$.  \end{center}

\end{proof}

\bigskip

\item \points{4} Prove that if $x$ is a real number such that $x^2+x>2$, then either $x<-2$ or $x>1$. (Hint:  use axioms and Theorems 1-2 in Notes 1)

\bigskip

\noindent {\bf Solution:} 

\begin{proof} Prove by contrapositive: If $x \geq -2$ and $x \leq 1$, then $x^2 + x \leq 2$.  

Since $x \geq -2$, then $x + 2 \geq 0$.  
Since $x \leq 1$, then $x - 1 \leq 0$. 

Recall the real number axiom: if $x \geq 0$ and $y \leq 0$, then $xy \leq 0$.  Thus $x + 2 \geq 0$ and $x - 1 \leq 0$.  

This implies that $(x + 2)(x - 1) \leq 0$.  Expanding $(x + 2)(x - 1)$, we get:
\begin{center} $(x + 2)(x - 1) = x^2 + x - 2 \leq 0$ \end{center}
\begin{center} $x^2 + x \leq 2$ \end{center}

Hence the contrapositive is true, then the original is also true.  

\end{proof}

\bigskip

\item \points{4} Prove that for every two positive real numbers $a$ and $b$ that

$$\left(a+b\right)\cdot \left(\frac{1}{a}+\frac{1}{b}\right)\ge 4.$$

(Hint:  use axioms and Theorems 1-2 in Notes 1)

\bigskip

\noindent {\bf Solution:} 

\begin{proof} First of all, lets rearrange the left side of the inequality:

$$\frac{a}{a} + \frac{a}{b} + \frac{b}{a} + \frac{b}{b} \geq 4$$

Let $a$ and $b$ be any positive real numbers.  Then, the inequality can be rewritten as:

$$\frac{a}{b} + \frac{b}{a} + 2 \geq 4$$ or $$\frac{a}{b} + \frac{b}{a} \geq 2$$

We know $\frac{a}{b} > 0$ and $\frac{b}{a} > 0$, so $\frac{a}{b} + \frac{b}{a} \geq 0$.  Then the equivalent inequality $$\frac{a^2 + b^2}{ab} \geq 0$$.  

Following from this: $$\frac{(a^2 + b^2 + 2ab) - 2ab}{ab} \geq 0$$

$$\frac{(a + b)^2}{ab} - \frac{2ab}{ab} \geq 0$$

$$\frac{(a + b)^2}{ab} \geq 2$$

Hence, $\frac{a}{b} + \frac{b}{a} \geq 2$

Therefore, for every two positive real numbers $a$ and $b$ that $\left(a+b\right)\cdot \left(\frac{1}{a}+\frac{1}{b}\right)\ge 4.$
\end{proof}

\bigskip

\item \points{4} Let $A, B, C$ be sets. Prove that $(A-B)\cup (A-C)=A-(B\cap C)$.

\bigskip

\noindent {\bf Solution:} 

\begin{proof} {\bf Step 1:} 
$(A - B) \cup (A - C) \subseteq A - (B \cap C)$

Let $x \in (A - B) \cup (A - C)$, then either $x \in (A - B)$ or $x \in (A - C)$.  
\begin{center} {\bf Case 1:} $x \in (A - B)$

Then $x \in A$ and $x \not\in B$

Since $x \not\in B$ then $x \not\in (B \cap C)$

Since $x \in A$ and $x \not\in (B \cap C)$

Then $x \in A - (B \cap C)$
\end{center}
\begin{center} {\bf Case 2:} $x \in (A - C)$

Then $x \in A$ and $x \not\in C$

Since $x \not\in C$ then $x \not\in (B \cap C)$

Since $x \in A$ and $x \not\in (B \cap C)$

Then $x \in A - (B \cap C)$. 
\end{center}

{\bf Step 2:} 
$A - (B \cap C) \subseteq (A - B) \cup (A - C)$
\begin{center}
Let $x \in A - (B \cap C)$

Then $x \in A$ and $x \not\in (B \cap C)$

Since, $x \not\in (B \cap C)$

Then $x \not\in B$ and $x \not\in C$

So, $x \in A \subseteq (A - B) \cup (A - C)$
\end{center}

{\bf Therefore}, $(A - B) \cup (A - C) = A - (B \cap C)$.  

\end{proof}


\item \points{4} Let $A$ and $B$ be sets. Prove that $A=(A-B)\cup (A \cap B)$. (Hint: if $x\in A$, then there are two cases: $x\in B$ or $x\not\in B$.)

\bigskip

\noindent {\bf Solution:} 

\begin{proof} {\bf Step 1:} Prove that $A \subseteq (A - B) \cup (A \cap B)$

Let $x \in A$, then $x \in B$ or $x \not\in B$.  
\begin{center} {\bf Case 1:} $x \in B$ 

Since $x \in A$ and $x \in B$, 

Then, $x \in (A \cap B) \subseteq (A - B) \cup (A \cap B)$
\end{center}
\begin{center} {\bf Case 2:} $x \not\in B$

Since $x \in A$ and $x \not\in B$,

Then, $x \in (A - B) \subseteq (A - B) \cup (A \cap B)$
\end{center}

{\bf Step 2:} Prove that $(A - B) \cup (A \cap B) \subseteq A$

Let $x \in (A - B) \cup (A \cap B)$.  

Then either $x \in (A - B)$ or $x \in (A \cap B)$.  
\begin{center}
{\bf Case 1:} $x \in (A - B)$

Then $x \in A$ and $x \not\in B$

and $x \in (A - B) \subseteq A$
\end{center}
\begin{center}
{\bf Case 2:} $x \in (A \cap B)$

then $x \in A$ and $x \in B$

So, $x \in (A \cap B) \subseteq A$.  
\end{center}

{\bf Therefore}, it is proven that $A = (A - B) \cup (A \cap B)$.  

\end{proof}

\item \points{extra 2} Prove that for every three positive real numbers $a$, $b$ and $c$ that

$$\left(a+b+c\right)\cdot \left(\frac{1}{a}+\frac{1}{b}+\frac{1}{c}\right)\ge 9.$$

\item \points{extra 2} Prove that for every three positive real numbers $a$, $b$ and $c$ that
    $$a^2+b^2+c^2\ge ab+bc+ac.$$
    
Note: For problem 4, 7 and 8, you can only use axioms and Theorems 1-2 in Notes 1, but not other more advanced theorems or known inequalities.

\end{enumerate}

\end{document}


