\documentclass[10pt]{article} %
\usepackage{fullpage}
\usepackage{graphicx}
\usepackage{graphics}
\usepackage{psfrag}
\usepackage{amsmath,amssymb}
\usepackage{amsthm}
\usepackage{amsfonts}
\usepackage{enumerate}

\setlength{\textwidth}{6.5in}
\setlength{\textheight}{9in}

\newcommand{\cP}{\mathcal{P}}
\newcommand{\N}{\mathbb{N}}
\newcommand{\Z}{\mathbb{Z}}
\newcommand{\R}{\mathbb{R}}
\newcommand{\Q}{\mathbb{Q}}
\newcommand{\points}[1]{{\it (#1 Points)}}
\newcommand{\tpoints}[1]{{\bf #1 Total points.}}

\title{Math 214 -- Foundations of Mathematics\\
Homework 6\\
{\large{\bf Due February 28, 2014}}}
\date{}
\author{Alexander Powell}


\begin{document}
\maketitle



\begin{enumerate}

%\item  \points{4}  \begin{enumerate}
 % \item Let $n\in \N$. Prove that either $n^2 \equiv 0$ (mod $4$) or $n^2 \equiv 1$ (mod $4$). (Hint: two cases of even and odd numbers)
 % \item Prove that the sum of two odd squares cannot be a square.
%\end{enumerate}

%\item \points{4}  Find the last digit of $2012^{2010}$.
%\item\points{4} Prove or disprove: Let $n\in \N$. Then $2n+1$ and $3n+2$ are relatively prime.

%\item\points{4} Prove for every integer $n$ that $n^2+1$ is not a multiple of $6$. (hint: every $n$ can be expressed by $6k+r$ where $0\le r<6$)


\item\points{4} \begin{enumerate}

                    \item Prove that $3|(n^3-n)$ for every  $n\in \N$;
                    
                    \bigskip
                    
                    \noindent {\bf Solution:} 
                    \begin{proof}If $3|(n^3-n)$ then there are $3$ cases: $n = 3k$, $n = 3k$, and $n = 3k$.  
                    
                    {\bf Case 1:} $n = 3k$
                    
                    $((3k)^3 - (3k))$
                    
                    $ = 27k^3 - 3k$
                    
                    $ = 3(9k^3 - k)$
                    
                    So, $3|(n^3-n)$ in this case.  
                    
                    {\bf Case 2:} $n = 3k + 1$
                    
                    $(3k+1)^3 - (3k + 1)$
                    
                    $ = 27k^3 + 27k^2 + 9k + 1 - 3k - 1$
                    
                    $ = 3(9k^3 + 9k^2 + 3k - k)$
                    
                    So, $3|(n^3-n)$ in this case.
                    
                    {\bf Case 3:} $n = 3k + 2$
                    
                    $(3k + 2)^3 - (3k + 2)$
                    
                    $ = 27k^3 + 54k^2 + 36k +8 - 3k - 2$
                    
                    $ = 3(9k^3 + 18k^2 + 12k - k + 2)$
                    
                    So, $3|(n^3-n)$ in this case.
                    
                    Therefore, $3|(n^3-n)$.  
                    \end{proof}
                    
                    
                    \item In Homework 3.6, we have proved that  $2|(n^3-n)$ for every  $n\in \N$. Use Theorem 11.16 and above results  to prove $6|(n^3-n)$ for every  $n\in \N$.
                    
                    \bigskip
                    
                    \noindent {\bf Solution:} 
                    \begin{proof} From theorem $11.16$, if $a|c$ and $b|c$ then $(ab)|c$ if $gcd(a,b) = 1$.  Since $2|(n^3 - n)$ from homework $3.6$ and $3|(n^3 - n)$ from the above solution, and $gcd(2,3) = 1$ then we can conclude that $(2 \cdot 3)|(n^3 - n)$ or $6|(n^3-n)$.  
                    \end{proof}
                  \end{enumerate}
                  
\item\points{4} Suppose that $a,b,c\in \N$. Prove that $gcd(a, c) =gcd(b, c) = 1$ if and only if $gcd(ab, c) = 1$.
(hint: Theorem 11.12: $gcd(a, b) = 1$ if and only if there exist $x, y \in \Z$ such
that $ax + by = 1$).

\bigskip

\noindent{\bf Solution:} 
\begin{proof}This must be proven with two steps:

{\bf Step 1:} We must prove that if $gcd(a,c) = gcd(b,c) = 1$ then $gcd(ab,c) = 1$.  From theorem $11.12$, $gcd(a,c) = 1$ iff $\exists x,y \in \Z$, such that:$$ax + cy = 1$$ and $$bw + cz = 1$$
Multiplying this out, we get:

$abxw = (1 - cy)(1 - cz)$

$abxw = 1 - c(y + Z -  cyz)$

$c(y + Z - cyz) + abxw = 1$

Therefore, $gcd(ab,c) = 1$

{\bf Step 2:} We must prove that if $gcd(ab,c) = 1$ then $gcd(a,c) = gcd(b,c) = 1$.  

So, $\exists x,y \in \Z$ such that $abx + cy = 1$, which can be rewritten as: $$b(ax) + cy = 1$$
Hence, there exist integer solutions to the equations $ax + cy = 1$ and $bw + cz = 1$.  This implies that $gcd(a,c) = gcd(b,c) = 1$.  
\end{proof}
\bigskip

\item\points{4} Suppose that $a,b,c,d\in \N$, and $gcd(a,b)=d$. Prove that if $a |c$ and $b|c$, then $(ab)|(cd)$.
(hint: use a similar proof as that of Theorem 11.16: since $d=ax+by$ for some $x,y\in \Z$, then $cd=c(ax+by)$)

\bigskip

\noindent{\bf Solution:}
\begin{proof} Since $gcd(a,b) = d$, then from theorem $11.12$, $\exists x,y \in \Z$ such that $$ax + by = d$$ 
Since $a|c$ and $b|c$, then $\exists d,e \in \Z$ such that $c = ad$ and $c = be$.  From $ax + by = d$ we get $$acx + bcy = dc$$
Substituting in the two equations for $c$ above we get:

$abex + bady = dc$

$a(bex) + b(ady) = dc$

$ab(ex + dy) = dc$

So, we can conclude that $(ab)|(cd)$.  
\end{proof}

\bigskip

\item\points{4} (prove directly, and do not use Theorem 11.20)
\begin{enumerate}
  \item Prove that $\sqrt{3}$ is irrational.
  
  \bigskip
  
  \noindent {\bf Solution:} 
  \begin{proof} To prove that $\sqrt{3}$ is irrational we will use a proof by contradiction.  Lets assume, to the contrary, that $\sqrt{3}$ is a rational number.  Then $\sqrt{3}$ can be represented by the ratio $\frac{a}{b}$ where $a,b \in \Z$.  So, 
  
  $3 = \frac{a^2}{b^2}$, or
  
  $3b^2 = a^2$
  
  Since $3|a^2$ and $3$ is prime, then from $11.14$, $3|a$ or $3|a$, thus $3|a$.  Also, $b^2 = \frac{a^2}{3}$ so $b^2$ is divisible by $3$, so $3|b$ or $3|b$, thus $3|b$.  However, if $a$ and $b$ are both divisible by $3$, that implies that they weren't in their lowest terms which is a contradtiction since we assumed that $gcd(a,b) = 1$.  Hence, we have that $\sqrt{3}$ is irrational.  
  \end{proof}
  \bigskip
  
  \item Prove that $\sqrt{6}$ is irrational. (Hint: you can use results from Homework 4 and Result 5.15 and Theorem 5.16)
  
  \bigskip
  
  \noindent {\bf Solution:}
  \begin{proof} Again, we will use a proof by contradition.  First, assume that $\sqrt{6}$ is rational, so that it can be expressed as $\frac{a}{b}$, where $a,b \in \Z, b \neq 0$, and $gcd(a,b) = 1$.  
  
  Like we did above, we can get the equation $6b^2 = a^2$
  
  $(2 \cdot 3)b^2 = a^2$
  
  Since $3|a^2$ and $3$ is prime, then $3|a$, so $\exists x \in \Z$ such that $a = 3x$.  So, we have:
  
  $(2 \cdot 3)b^2 = (3x)^2 = 9x^2$
  
  $2b^2 = 3x^2$
  
  Since $3|(2b^2)$ and $gcd(3,2) = 1$ then from theorem $11.13$ we get $3|b^2$ and since $3|b^2$ and $3$ is prime then from $11.14$ we know $3|b$.  So, now we have found that $3|a$ and $3|b$ so $3$ is a common divisor of $a$ and $b$.  Therefore, $gcd(a,b) \geq 3$ which contradicts with $gcd(a,b) = 1$.  Hence $\sqrt{6}$ is irrational.  
  \end{proof}
  
  \bigskip
  
\end{enumerate}

    \item\points{4}  It is known from Theorem 11.20 that $\sqrt{k}$ is an irrational number if $k\in \N$ and $k\ne m^2$ for some $m\in \N$. Let $x\in \R$ and $x>0$. Prove that there exists an irrational number between $0$ and $x$. (hint: there are two cases: $x$ is rational, or $x$ is irrational)
    
    \bigskip
    
    \noindent{\bf Solution:}
    \begin{proof} This will be a proof by cases: $x$ is rational, or $x$ is irrational.  
    
    {\bf Case 1:} $x \in \mathbb{I}$ (Irrational)
    
    If $x$ is irrational, then the proof is simple:
    
    Let $y = \frac{1}{2} x$, We know from Result $5.15$ that an irrational number ($x$) multiplied by a rational number($\frac{1}{2}$) is irrational.  Therefore, $y$ is an irrational number between $0$ and $x$.  
    
    {\bf Case 2:} $x \in \mathbb{Q}$ (Rational)
    
    For the second case we can take a number like $\frac{\sqrt{2}}{2}$ which is between $0$ and $1$ and is irrational.  Then there exists the number $y = \frac{\sqrt{2}}{2} x$ which is between $0$ and $x$ and makes $y$ an irrational number.  
    \end{proof}
    
    \bigskip

    \item\points{4}
    \begin{enumerate}
  \item Express each of the integers $4278$ and $71929$ as  a product of primes (canonical factorization).
  
  \bigskip
  
  \noindent {\bf Solution:} The canonical factorization of $4278$ can be expressed as $1^1 \times 2^1 \times 3^1 \times 23^1 \times 31^1$, and the canonical factorization of $71929$ can be expressed as $1^1 \times 11^1 \times 13^1 \times 503^1$.  
    
  \bigskip
  
  \item Find $gcd(4278,71929)$ by using the canonical factorization.
  
  \bigskip
  
  \noindent {\bf Solution:} Because the canonical factorizations of the two numbers share nothing in common except for $1$, the $gcd(4278,71929) = 1^1$ or $1$.  
  
  \bigskip
  
\end{enumerate}

%\item\points{4} Suppose that $a,b,c\in \N$, and $gcd(b,c)=1$. Prove that if $b |a$ and $c|a$, then $(bc)|a$.

 \item\points{extra 2} Prove that if $p$ and $q$ are prime numbers with $p\ge q\ge 5$, then $24|(p^2-q^2)$.

\item\points{extra 2} Let $n\in \N$ and let $\displaystyle n=\sum_{i=0}^k a_i \cdot 10^i$ be its decimal expression where $a_i\in \{0,1,2,3,4,5,6,7,8,9\}$. Find a necessary and sufficient condition on $\{a_i\}$ so that $7|n$, and prove it. (Indeed problem 11.66 in textbook at least gives a special case, so you may generalize that one.)

\end{enumerate}
%\item (extra 2 points) Find the last three digits of $7^{9999}$.


\end{document}



%%% Local Variables:
%%% mode: latex
%%% TeX-master: t
%%% End:
