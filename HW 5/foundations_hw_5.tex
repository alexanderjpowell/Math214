\documentclass[10pt]{article} %
\usepackage{fullpage}
\usepackage{graphicx}
\usepackage{graphics}
\usepackage{psfrag}
\usepackage{amsthm}
\usepackage{amsfonts}
\usepackage{amsmath,amssymb}
\usepackage{enumerate}

\setlength{\textwidth}{6.5in}
\setlength{\textheight}{9in}

\newcommand{\cP}{\mathcal{P}}
\newcommand{\N}{\mathbb{N}}
\newcommand{\Z}{\mathbb{Z}}
\newcommand{\R}{\mathbb{R}}
\newcommand{\Q}{\mathbb{Q}}
\newcommand{\points}[1]{{\it (#1 Points)}}
\newcommand{\tpoints}[1]{{\bf #1 Total points.}}

\title{Math 214 -- Foundations of Mathematics\\
Homework 5\\
{\large{\bf Due February 20, 2014}}}
\date{}
\author{Alexander Powell}


\begin{document}
\maketitle

Solve the following problems.  Please remember to use complete sentences
and good grammar.

\begin{enumerate}



\item \points{4} Use proof by contradiction in both parts:

\begin{enumerate}
  \item Prove that the product of an irrational number and a nonzero rational number is irrational.
  
  \bigskip
  
  \noindent {\bf Solution:}
  \begin{proof} In symbols, the statement we are trying to prove can be written as:
  $$\forall x \in \mathbb{I}, \forall y \in \mathbb{Q}, y \not = 0, xy \in \mathbb{I}$$
  
  To prove by contradition we need to find the negation of this statement, which can be written:
  $$\exists x \in \mathbb{I}, \exists y \in \mathbb{Q}, y \not = 0, xy \in \mathbb{Q}$$
  
  Now this is simply an existence proof.  Let $x$ be $\pi$ (an irrational number) and let y be $1$ (a ratioal number).  
  
  Then the product of the two can be expressed as:
  $$\pi \cdot 1 = \pi,$$ which of course is irrational.  This is a contradictions of the negation of the original statement.  Therefore, the original statement, is true.  
  \end{proof}
  
  \bigskip
  
  \item Prove that there is no smallest positive irrational number.
  
  \bigskip
  
  \noindent {\bf Solution:}
  \begin{proof} This will be a proof be contradictions.  To the contrary, suppose there is a smallest positive irrational number, say $y$.  Then for every positive irrational number $x$ we have $x \geq y$.  But if we take $x = y/2$, then $x$ is a positive and irrational number but $x < y$.  This contradicts our assumption that $y$ is the smallest positive irrational number, that is, $x \geq y$.  For every positive irrational number $x$.  Hence, there is no smallest positive  irrational number.  
  \end{proof}
  
  \bigskip
  
\end{enumerate}

\item \points{4} Prove by contradiction that if $a$ and $b$ are odd integers, then $4 \not | (a^2+b^2)$.

\bigskip

\noindent {\bf Solution:}
\begin{proof} To prove by contradiction, assume the negation is true, that is, $a$ and $b$ are odd integers and $4|(a^2 + b^2)$.  

Let $a$ and $b$ be odd integers, then $a = 2n+1, n\in \Z$ and $b = 2m+1, m \in \Z$.  

So, $(2n+1)^2 + (2m+1)^2$

$ = 4n^2 + 4n + 1 + 4m^2 + 4m + 1$

$ = (4m^2 + 4n + 4m^2 + 4m + 4) - 2$

$ = 4(n^2 + n + m^2 + m + \frac{1}{2}) - 2$

Since $4$ does not divide $(a^2 + b^2)$ we have a contradiction.  Hence the original statement is true.  
\end{proof}

\bigskip

\item \points{4} Consider any three consecutive positive integers. Prove that the cube of the largest cannot be the sum of the cubes of the other two.

\bigskip

\noindent {\bf Solution:}
\begin{proof} This will be a proof by contradiction.  The original statement can be rewritten as: "For any three consecutive integers, the cube of the largest is never equal to the sum of the cubes of the other two."  Now suppose the negation of the original statement which can be expressed as: "There exist three consecutive integers such that the cube of the largest is the sum of the cubes of the other two."  

Let the three consecutive integers be expressed as: $n, n+1, and n+2$.  So, 
\begin{center} $(n+2)^3 = (n+1)^3 + n^3$ \end{center}

\begin{center} $n^3 + 6n^2 + 12n + 8 = n^3 + 3n^2 + 3n + 1 + n^3$ \end{center}

\begin{center} $-n^3 + 3n^2 + 9n + 7 = 0$ \end{center}

There is only one solution to this equation, $n = 5.0546$, which is not a positive integer.  This is a contradiction.  Hence, the original statement that for any three consecutive positive integers, the cube of the largest cannot be the sum of the cubes of the other two.

\end{proof}

\bigskip

\item\points{4} Show that, except for $2$ and $5$, every prime can be expressed as $10k+1$, $10k+3$, $10k+7$ or $10k+9$, where $k\in \Z$.

\bigskip

\noindent {\bf Solution:} 
\begin{proof} Let all prime numbers, except $2$ and $5$ be represented with the letter $P$.  Appllying theorem 11.4, we have: 
\begin{center}$\exists q \in \N \cup \{0\}$ \end{center}
\begin{center} and $0 \leq r < 10$ such that $P = 10k+r$, then $r$ can be $0,1,2,3,4,5,6,7,8,$ or $9$. \end{center}
If $r = 0$ then $P = 10k$ which is not prime. 

If $r = 2$ then $P = 10k+2 = 2(5k+1)$ which is not prime. 

If $r = 4$ then $P = 10k+4 = 2(5k+2)$ which is not prime. 

If $r = 5$ then $P = 10k+5$ which includes the prime number 5. 

If $r = 6$ then $P = 10k+6 = 2(5k+3)$ which is not prime. 

If $r = 8$ then $P = 10k+8 = 2(5k+4)$ which is not prime. 

These are all contradictions.  Hence, $r$ can only be $1,3,7,$ or $9$.  Therefore, $P$ must be of form $10k+1$, $10k+3$, $10k+7$ or $10k+9$.  

\end{proof}

\bigskip

\item\points{4}
Find $\gcd(51,288)$ by using the Euclidean Algorithm, and find $m, n \in \Z$ such that $\gcd(51,288) = 51n + 288m$ (Show your  intermediate quotients and remainders).

\bigskip

\noindent {\bf Solution:} $$\gcd(51,288)$$
$$288 = 51 \cdot 5 + 33$$
$$51 = 33 \cdot 1 + 18$$
$$33 = 18 \cdot 1 + 15$$
$$18 = 15 \cdot 1 + 3$$
$$15 = 15 \cdot 3 + 0$$

So, the $\gcd(51,288) = 3$.  

Now, $51n + 288m = 3$, can be expressed as $m = 17k + 14, n = -96k - 79, k \in \Z$.  

\bigskip

\item\points{4} Prove or disprove: Let $n\in \N$. Then $2n+1$ and $3n+2$ are relatively prime.

\bigskip

\noindent {\bf Solution:}
\begin{proof} We will use theorem 11.9 to prove that for $n\in \N$, $2n+1$ and $3n+2$ are relatively prime.  Let $3n+2 = (2n+1) \cdot + (n+1)$.  From 11.9 we have: $$\gcd(3n+2,2n+1) = \gcd(2n+1,n+1)$$
Applying theorem 11.4 we get: $$(2n+1) = (n+1) \cdot 1 + n$$ and then $$(n+1) = n \cdot 1 + 1$$

Also, we can say $\gcd(2n+1,n+1) = \gcd(n+1,n) = gcd(n,1)$

Therefore, it is proven that if $n \in \Z, 2n+1$ and $3n+2$ are relatively prime.  
\end{proof}

\bigskip

%\item\points{4} Prove that every two consecutive odd positive integers are relatively prime.

\item \points{Extra 4} Prove that there is no rational number solution to the equation $x^2-3x+1=0$.
    (Note: we do not assume that we know all the solutions of $x^2-3x+1=0$ are given by quadratic formula)
%\item \points{4} Disprove the statement: There exists a real number $x$ such that $x^6+x^4+1=2x^2$.

%\item \points{Extra 4} If $p$ is a prime number larger than $5$, then $24 | (p^2 - 1)$.
\end{enumerate}

\end{document}


