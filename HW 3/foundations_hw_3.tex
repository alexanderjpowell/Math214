\documentclass[10pt]{article} %
\usepackage{fullpage}
\usepackage{graphicx}
\usepackage{graphics}
\usepackage{psfrag}
\usepackage{amsmath,amssymb}
\usepackage{enumerate}
\usepackage{amsthm}

\setlength{\textwidth}{6.5in}
\setlength{\textheight}{9in}

\newcommand{\cP}{\mathcal{P}}
\newcommand{\N}{\mathbb{N}}
\newcommand{\Z}{\mathbb{Z}}
\newcommand{\R}{\mathbb{R}}
\newcommand{\Q}{\mathbb{Q}}
\newcommand{\C}{\mathbb{C}}
\newcommand{\points}[1]{{\it (#1 Points)}}
\newcommand{\tpoints}[1]{{\bf #1 Total points.}}

\title{Math 214 -- Foundations of Mathematics\\
Homework 3\\
\author{Alexander Powell}
{\large{\bf Due February 5, 2014}}}
\date{}


\begin{document}
\maketitle

Solve the following problems. For proof problems, please remember to use complete sentences
and good grammar.

\begin{enumerate}
\item \points{4} For parts (a) - (c), first, write the statements so that there are no $\sim$ symbols. Then, rewrite the statements in words so that there are no $\forall, \exists, \in$ or $=$ symbols. Finally determine whether the statement is true or false. 
\begin{enumerate}
\item $\sim(\forall x\in \R, \exists y\in \R, xy=1)$;

\bigskip

\noindent {\bf  Solution:} $(\exists x\in \R, \forall y\in \R, xy \neq 1)$

There exists some real number $x$, such that for all real numbers $y$, the product of $x$ and $y$ does not equal $1$.  This statement is {\bf True}.  

\bigskip

\item $\sim(\exists y\in \R, \forall x\in \R, xy=0)$;

\bigskip

\noindent {\bf  Solution:} $(\forall y\in \R, \exists x\in \R, xy \neq 0)$

For all real numbers $y$, there exists some real number $x$ such that the product of $x$ and $y$ does not equal $0$.  This statement is {\bf False}.  

\bigskip

\item $\sim(\exists n\in \Z, \exists m\in \Z, m\le n)$.

\bigskip

\noindent {\bf  Solution:} $(\forall n\in \Z, \forall m\in \Z, m > n)$

For all integers $n$, and for all integers $m$, $m > n$.  This statement is {\bf False}.  

\bigskip

\end{enumerate}

\item \points{4}\begin{enumerate}
        \item \lq\lq Some birds sing all the time; 	All birds sing sometimes.\rq\rq\\
  The first part of sentence can be  quantified as: \lq\lq $\exists$ bird $b$ such that, $\forall$ time $t$, $b$ sings.\rq\rq\ Use a similar quantified statement for the second part.
  
  \bigskip
  
  \noindent {\bf  Solution:} $\exists$ time $t$ such that, $\forall$ bird $b$, $b$ sings.  
  
  \bigskip

      \item Consider the sentence, \lq\lq For every integer $n>0$ there exists some real number $x>0$ such that $x<1/n$.\rq\rq\ Without using words of negation, write a complete sentence that negates the sentence. Which sentence (the original or the negation) is true?
      
      \bigskip
      
      \noindent {\bf  Solution:} There exists an integer $n \leq 0$ such that for all real numbers $x \leq 0$, $x \geq 1/n$.  The original sentence is {\bf True}.  
      
      \bigskip
      
      \end{enumerate}





\item \points{2} Prove directly that if $x$ is an even integer, then $7x+5$ is an odd integer.

\bigskip

\noindent {\bf  Solution:} 

\begin{proof}
Let $x$ be an even number, then there exists $m\in \Z$ such that $x = 2m$.  So, 

$7x + 5 = 7 (2m) + 5$

$= 14m + 5$

$= 2(7m + 2) + 1$

Which is an odd integer.  

\end{proof}

\bigskip

 \item \points{2} Prove by contrapositive that if $3x+5$ is even then $x+2$ is odd.
 
 \bigskip
 
 \noindent {\bf Solution:} 
 
 \begin{proof}
 First, prove contrapositive: If $x + 2$ is even, then $3x + 5$ is odd.  
 
 Let $x + 2$ be even, then $\exists n\in \Z$ such that 
 
 $x + 2 = 2n$ or $x = 2n - 2$
 
 So, $3x + 5 = 3 (2n - 2) + 5$
 
 $ = 6n - 6 + 5$
 
 $ = 6n - 1$
 
 $ = 2 (3n) - 1$
 
 Which is an odd number.  Hence the contrapositive is true then the original is also true.  
 
 \end{proof}
 
 \bigskip

 \item \points{4} Let $x\in \Z$. Use a lemma to prove that if $7x-4$ is even, then $3x-11$ is odd. (Hint: $x$ should be even or odd?)
 
 \bigskip
 
 \noindent {\bf Solution:} 
 
 \begin{raggedright}
 {\bf Step 1:} If $7x - 4$ is even, then $x$ is even.  
 \end{raggedright}
 \begin{proof}
 
 {\bf Use contrapositive:}  If $x$ is odd, then $7x - 4$ is odd.  
 
 Let $x$ be odd, then $x = 2n + 1$ for some $n\in \Z$.  So, 
 
 $7x - 4 = 7 (2n + 1) - 4$
 
 $ = 14n + 7 - 4$
 
 $ = 14n + 3$
 
 $ = 2 (7n + 1) + 1$
 
 Which is an odd number.  Hence the contrapositive is true, the original is also true.  
 
 \end{proof}
 
 \begin{raggedright}
 {\bf Step 2:} If $x$ is even, then $3x - 11$ is odd.  
 \end{raggedright}
 \begin{proof}
 Let $x$ be even, then $x = 2n$ for some $n\in \Z$.  
 
 $3x - 11 = 3 (2n) - 11$
 
 $ = 6n - 11$
 
 $ = 6n - 10 - 1$
 
 $ = 2 (3n - 5) - 1$
 
 Which is an odd number.  Therefore, it is proven that if $7x - 4$ is even, then $3x - 11$ is odd.  
 
 \end{proof}
 
 \bigskip

\item \points{4} Prove that if $n\in \Z$, then $n^3-n$ is even. (Hint: use proof by cases)

\bigskip

\noindent{\bf Solution:}  

There are two cases to this proof: either $n$ is even or odd.  

\begin{raggedright} {\bf Case 1:} If $n$ is even. \end{raggedright}

\begin{proof}
Let $n$ be even, then $n = 2x$ for some $x \in \Z$.  

$n^3 -n = (2x)^3 - 2x$

$ = 8x^3 - 2x$

$ = 2 (4x^3 - x)$

Which is an even number.  
\end{proof}

\begin{raggedright} {\bf Case 2:} If $n$ is odd. \end{raggedright}

\begin{proof}
Let $n$ be odd, then $n = 2x + 1$ for some $x \in \Z$.  

$n^3 - n = (2x + 1)^3 - (2x + 1)$

$ = (4x^2 + 4x + 1)(2x + 1) - (2x + 1)$

$ = 8x^3 + 8x^2 + 2x + 4x^2 + 4x + 1 - 2x - 1$

$ = 8x^3 + 12x^2 + 4x$

$ = 2(4x^3 + 6x^2 + 2x)$

Which is an even number.  

\end{proof}

\begin{raggedright} Therefore, it is proven that if $n\in \Z$, then $n^3 - n$ is even.  \end{raggedright}

\bigskip

\item \points{4} Let $a,b\in \N$. Prove that if $ab=4$, then $(a-b)^3-9(a-b)=0$.

\bigskip

\noindent{\bf Solution:} 

This is a proof by cases.  

\begin{raggedright} {\bf Case 1:} $a = 2, b = 2$ \end{raggedright}

\begin{proof}
Let $a = b = 2$, then

$(2 - 2)^3 - 9(2 - 2) = 0$

$0 - 9(0) = 0$

This satisfies the hypothesis.  

\end{proof}

\begin{raggedright} {\bf Case 2:} $a = 1, b = 4$ \end{raggedright}

\begin{proof}
Let $a = 1$ and $b = 4$, then

$(1 - 4)^3 - 9(1 - 4) = 0$

$-27 - 9(-3) = 0$

This satisfies the hypothesis.  

\end{proof}

\begin{raggedright} {\bf Case 3:} $a = 4, b = 1$ \end{raggedright}

\begin{proof}
Let $a = 4$ and $b = 1$, then

$(4 - 1)^3 - 9(4 - 1) = 0$

$27 - 9(3) = 0$

This satisfies the hypothesis.  

\end{proof}

\begin{raggedright}  Therefore, it is proven that if $ab = 4$, then $(a-b)^3-9(a-b)=0$.  \end{raggedright}

\bigskip

\item \points{extra 4} It was proved by Andrew Wiles and Richard Taylor that Fermat Theorem is true: for every $x,y,z\in \N$ and $n\in \N$, $n\ge 3$, $x^n+y^n\ne z^n$. Here we consider the case when $n$ is a negative integer.
    \begin{itemize}
      \item Show that there are infinitely many $x,y,z\in \N$ such that $$\frac{1}{x}+\frac{1}{y}=\frac{1}{z}.$$
      \item Prove or disprove that there are infinitely many $x,y,z\in \N$ such that $$\frac{1}{x^2}+\frac{1}{y^2}=\frac{1}{z^2}.$$
      \item Assuming the Fermat Theorem, show that there is no $x,y,z\in \N$ and $n\ge 3$ such that $$\frac{1}{x^n}+\frac{1}{y^n}=\frac{1}{z^n}.$$
    \end{itemize}
%\item \points{4} Let $A, B, C$ be sets. Prove that $(A-B)\cup (A-C)=A-(B\cap C)$.







%   \newcommand{\ceil}[1]{\rceil #1 \lceil}
% \item \points{4} For a real number, $\alpha$, define the ceiling of
%   $\alpha$, written as $\ceil{\alpha}$, as the minimum integer $n$
%   such that $n \geq \alpha$, i.e., $\ceil{alpha} = \min\{n \in \N : n
%   \geq \alpha\}$. Show that

%\item \points{4} Let $A \subset \R$ be a well-ordered set.  Prove that if $B \subset A$ then $B$ is well-ordered.



%\item \points{extra 4} The set of complex numbers is $\C=\{a+bi:a,b\in \R\}$ where $i$ satisfies $i^2=-1$. One can prove that $\C$ is a field (in the same sense as the field definition given in notes). Prove that there is no any order relation can be defined on $\C$ so $\C$ is an ordered field. (all definitions are in the notes)

\end{enumerate}

\end{document}


